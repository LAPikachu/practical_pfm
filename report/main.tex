
\input{preamble.tex}

\begin{document}

%
\begin{titlepage}
\begin{center}
\includegraphics[width=0.5\textwidth]{graphics/FAU_TechFak_EN_H_black.eps}

\LARGE Department Materials Science

\Large WW8: Materials Simulation

\LARGE \textbf{Practical: Phase-Field Method}

\Large Basics and Application in Materials Science



\vfil
\Large Leon Pyka (22030137)



\Large \textbf{Supervision: Dr. Frank Wendler }
\end{center}

\thispagestyle{empty}
%
\end{titlepage}
%

\setcounter{page}{1}

\tableofcontents
\newpage
\section{Introduction}
Phase-field simulation is a versatile application in the toolbox of materials simulation. It is often used for simulations of phase transitions, dislocation evolution, fracture simulations etc. The following practicals aim is to get a practical introduction into the subject. In two separate tasks a 1D single-component solidification simulation, calculation of a bulk energy density coefficient and gradient energy density coefficient are going to be conducted. 

\section{Task 1: 1D Single-Component Solidification} \label{sec:task1}

To simulate a 1D single-component solidification we utilize the following energy density:

\begin{equation}
	F = \int \bigl( f_{0} \phi^{2}(1- \phi)^{2}  + \frac{K_{\phi}}{2} \lvert \nabla \phi \rvert ^{2} \bigr) d \vec{r}
\end{equation}

As solidification is a a non-conservative process, the kinetics are governed by the Allen-Cahn equation (for the  homogeneous, isotropic case).

\begin{equation}
	\frac{\partial \phi}{\partial t} =-L \frac{\delta F}{\delta \phi} \label{eq:alle_cahn_homo_iso}
\end{equation}

Here the functional derivative \(\frac{\delta F}{\delta \phi}\) of an energy functional eq. \ref{eq:energy_functional_general} can be evaluated as eq. \ref{eq:func_derivative_general}.\\


\begin{subequations}
	\begin{align}
		F =& \int f(\vec{r}, \phi, \nabla \phi) d\vec{r} \label{eq:energy_functional_general} \\
		\frac{\delta F}{\delta \phi} =& \frac{\partial f}{ \partial \phi} - \nabla \cdot \frac{\partial f}{\partial (\nabla \phi)} \label{eq:func_derivative_general}
		\end{align}
\end{subequations}

Solving the functional derivative from eq. \ref{eq:alle_cahn_homo_iso} yields:

\begin{subequations}
	\begin{align}
		\frac{\delta F}{\delta \phi} =& 2 f_{0}\phi (1-\phi)^2 + 2 f_{0}\phi^{2} (\phi -1) - K_{\phi} \nabla^{2} \phi \Leftrightarrow \\
		\Leftrightarrow  & 2\phi^{3}  -4\phi^{2} + 2\phi - K_{\phi} \nabla^{2} \phi
	\end{align}
\end{subequations}

Inserting the result into eq. \ref{eq:alle_cahn_homo_iso} results in:

\begin{equation}
	\frac{\partial \phi}{\partial t} = -L \bigl[  f_0 ( 2\phi^{3} -4\phi^{2} + 2\phi) - K_{\phi} \nabla^{2} \phi \bigr]
\end{equation}

As the governing equation for solving the PDE we apply Neumann boundary conditions for the left \(\phi_{0} = \phi_{1}\) and the right side \(\phi_{n}= \phi_{n-1}\) and discretize the partial derivative \(\frac{\partial \phi}{\partial t}\) as:

\begin{subequations}
	\begin{align}
	\frac{\partial \phi}{\partial t}  \approx \frac{\phi_{next} - \phi}{\Delta t} &=  -L \bigl[  f_0 ( 2\phi^{3} -4\phi^{2} + 2\phi) - K_{\phi} \nabla^{2} \phi \bigr] \Leftrightarrow \\
	\Leftrightarrow \phi_{next} & = \phi -\Delta t L \bigl[  f_0 ( 2\phi^{3} -4\phi^{2} + 2\phi) - K_{\phi} \nabla^{2} \phi \bigr]
    \end{align}
\end{subequations}

and \(\nabla^{2} \phi(x) \) is discretized according to the (central) finite differences scheme:

\begin{equation}
	\nabla^{2} \phi(x_{i}) \approx \frac{\phi(x_{i+1}) - 2\phi(x_{i}) + \phi(x_{i-1}) }{\Delta x^{2}} \label{eq:central_euler_second_derivative}
\end{equation}

where \(x_{i}\) is the node at which the value is computed and \(\Delta x\) is the node distance. The initial result is plotted in fig. \ref{fig:solidification_init}. As can be seen in fig. \ref{fig:solidification_f0} with decreasing \(f_{0}\) the total energy decreases faster over time. With decreasing value the transition region (where \(\phi\) changes from 0 to 1) becomes broader. The opposite effect can be observed in fig. \ref{fig:solidification_K} with variation of \(K_{\phi}\). Additional the energy density functional seems to become broader but only changes its peak value for increased K-values. L does not seem to have any influence on the \(\phi\)-field, the energy or energy density - see fig. \ref{fig:solidification_L} .


\begin{figure}[htb]
	\centering
	\includegraphics[width=\textwidth]{solidification_initial.png}
	\caption{Initial results of the 1D single-component solidification simulation.}
	\label{fig:solidification_init}
\end{figure}

\begin{figure}[htb]
	\centering
	\includegraphics[width=\textwidth]{solidification_vary_f0.png}
	\caption{Results of the 1D single-component solidification simulation with various \(f_{0}\) values, with constant values K=0.1 and L = 1.}
	\label{fig:solidification_f0}
\end{figure}

\begin{figure}[htb]
	\centering
	\includegraphics[width=\textwidth]{solidification_vary_K.png}
	\caption{Results of the 1D single-component solidification simulation with various K values with constant values \(f_0 \) =1 and L = 1.}
	\label{fig:solidification_K}
\end{figure}


\begin{figure}[htb]
	\centering
	\includegraphics[width=\textwidth]{solidification_vary_L.png}
	\caption{Results of the 1D single-component solidification simulation with various L values with constant values K = 0.1 and \(f_0\) = 1.}
	\label{fig:solidification_L}
\end{figure}

\section{Task 2: Determination of \(f_{0}\)}
The equilibrium Al concentration of the disordered \(\gamma \)-Phase \(c_{\gamma}^{e}\) has been determined to be 0.165 and the  Al concentration of the ordered \(\gamma ' \)-Phase \(c_{\gamma '}^{e}\) has been determined to be 0.230 - which is shown in fig. \ref{fig:AlNi_gibbs} \cite{zhu2002}. This however clashes with the value that is shown in the problem statement \cite{zaisera} - which features the following fig. \ref{fig:AlNi_problem_statement}. For reasons of consistency we will assume \(c_{\gamma}^{e}\) to be 0.16 and \(c_{\gamma '}^{e}\) to be 0.23. The bulk energy density function is constructed as:

\begin{equation}
	f_{bulk} = f_{0} (c_{\gamma '}^{e} - c)^{2}(c_{\gamma}^{e} - c)^{2} \label{eq:f_bulk_NiAl}
\end{equation}

The \( f_{0}\) needs to be fitted so that it results in an energy barrier of \(\Delta f = 149 \frac{\mathrm{J}}{\mathrm{mol}} \) \cite{zaisera}. This is accomplished by choosing the right \(f_{0}\) to reach the right value at the local maximum of eq. \ref{eq:f_bulk_NiAl}. The local maximum of the function can be found automatically by finding the second root of the derivative of eq. \ref{eq:f_bulk_NiAl}:

\begin{equation}
	\frac{df_{bulk}}{dc} = -2 f_{0} (c_{\gamma '}^{e} - c)(c - c_{\gamma}^{e})^{2} + 2 f_{0} (c_{\gamma '}^{e} - c)^{2}(c - c_{\gamma}^{e})
\end{equation}

Factoring out \(f_{0} (c_{\gamma '}^{e} - c)(c - c_{\gamma}^{e})\) to find the root of the function we are left with:

\begin{equation}
	c_{max} = \frac{1}{2} (c_{\gamma}^{e} + c_{\gamma '}^{e})
\end{equation}

 Here \(c_{max}\) is found at 0.195. Solving \ref{eq:f_bulk_NiAl} for \(f_{0}\) results in a value of \(\approx 100.9 \cdot 10^{6} \frac{\mathrm{J}}{\mathrm{mol}}\).  The resulting function of the corresponding parameters is plotted in \ref{fig:alni_fitted}.


\begin{figure}[htb]
	\centering\includegraphics[width=0.5\textwidth]{NiAl_gibbs_free_energy.png}
	\caption{Calculated chemical free energy as a function of composition for both the disordered and ordered phase at 1300 K reproduced from \cite{zhu2002}.}
	\label{fig:AlNi_gibbs}
\end{figure}

\begin{figure}[htb]
	\centering\includegraphics[width=0.5\textwidth]{alni_bulk_denergy_density_problem_statement.png}
	\caption{Bulk energy density as a function of concentration of Al at 1300 K reproduced from \cite{zaisera}.}
	\label{fig:AlNi_problem_statement}
\end{figure}

\begin{figure}[htb]
	\centering
	\includegraphics[width=\textwidth]{alni_fitted.png}
	\caption{Plot of bulk energy of binary NiAl mixture with \(c_{\gamma}^{e} = 0.165\),  \(c_{\gamma '}^{e} = 0.23\) and \(f_{0} = 133.6 \cdot 10^{6} \frac{\mathrm{J}}{\mathrm{mol}} \).}
	\label{fig:alni_fitted}
\end{figure}



\section{Task 3: Determination of \(K_{c}\) }
In this task the gradient energy density coefficient of a binary NiAl-system is calculated. Apart from the governing equation being the Cahn-Hillard equation (eq. \ref{eq:cahn_hillard} ) as we are dealing with a conserved field, the general algorithm is similar to sec. \ref{sec:task1}. 

\begin{equation}
	\frac{ \partial c}{ \partial t} = M \nabla^{2} \frac{\delta F}{\delta c} \label{eq:cahn_hillard}
\end{equation} 

The functional for the binary system is evaluated as follows:

\begin{equation}
	\frac{\delta F}{\delta c} = 2 f_{0}(c_{\gamma '} - c )(c - c_{\gamma})^2 + 2 f_{0} (c - c_{\gamma})(c_{\gamma '} - c )^{2} - K_{c} \nabla^{2} c  \label{eq:functional_NiAl}
\end{equation}
		
\(\nabla^{2}(\cdot)\) is (both in eq. \ref{eq:functional_NiAl} and eq. \ref{eq:cahn_hillard}) is discretized as in sec. \ref{sec:task1} eq. \ref{eq:central_euler_second_derivative}.  The following coefficients are used molar Volumen \(V_{m} = 10^{-5} \frac{m^{3}}{mol}\), interface mobility \(M = 10^{-17} \frac{mol^{2}}{J m s}\), length segment \(dx = 10^{-8} m\), \(f_{0} = 134 \cdot 10^{6} \frac{J}{mol}\) and a experimental interface energy of \( F_{inte} = 5 \sim 50 \cdot 10^{-3} \frac{J}{m}\). 

As M is function of molar quantity of substance for calculating the energy we have to define an adjusted \( M_{new} = M V_{m}^{2}\). Running the simulation for various values of \(K_{c}\) results in fig. \ref{fig:vary_K}. The total energies F are calculated, by applying the \code{numpy.trapz} method - integrating \(f_{inte}\) over x.  Printing the energies alongside the \(K_{c}\) values shows that \( 1. 0 \cdot 10^{-6} < K_{c} < 1.0 \cdot 10^{-7} \frac{J}{m}\) produced results in the experimentally observed Range of \( 0.05 < F_{inte} < 0.5 \frac{J}{mol}\).

\begin{figure}[htb]
	\includegraphics[width=\textwidth]{find_K_grad_energy_density.png} 
	\caption{Results of the conserved binary NiAl-system simulation with various \(K_{c}\) values.}
	\label{fig:vary_K}
\end{figure}

For expanding the calculation of the concentrations we need to implement eq. \ref{eq:central_euler_second_derivative} in 2D:

\begin{equation}
	\nabla^{2} c_{i,j} = \frac{c_{i,j-1} + c_{i,j+1} - 4c_{i, j} + c_{i-1, j} + c_{i+1, j}}{\Delta x^{2}} 
\end{equation}

where the concentrations are mapped on a \(n \times n\)-Matrix. Running the simulation for 1000\(dt\) results in fig. \ref{fig:NiAl_2D_heatmaps} and . Here we applied an initially random concentration with \(c_{init} = 0.195\) as a mean value for a gauss distributed matrix using the \code{numpy.random.normal(loc=0.195, scale=0.1, size=(N,N))} function. Choosing unfitting K values (i.e.\(3 \cdot ^{-5}\)) we see that the formation of domains does not occur - fig. 
 \ref{fig:NiAl_2D_heatmaps_K3e-5}. While for the K value \(1 \cdot ^{-6}\))   over time we see, the development of several domains of \(\gamma \\ \gamma '\)-Phases - fig. \ref{fig:NiAl_2D_heatmaps}. For smaller values like \(10^{-7}\) the domains become thinner, and less pronounced \ref{fig:NiAl_2D_heatmaps_K1e-7}.

\begin{figure}[htb]
	\includegraphics[width=\textwidth]{NiAl_2D_K_3e-5.png} 
	\caption{Heatmaps of concentrations over time calculated by conserved phase-field method from an initially randomly distributed concentration map with a K value of \(3 \cdot ^{-5}\).}
	\label{fig:NiAl_2D_heatmaps_K3e-5}
\end{figure}

\begin{figure}[htb]
	\includegraphics[width=\textwidth]{NiAl_2D_K1e-6.png} 
	\caption{Heatmaps of concentrations over time calculated by conserved phase-field method from an initially randomly distributed concentration map with a K value of \(10^{-6}\).}
	\label{fig:NiAl_2D_heatmaps}
\end{figure}

\begin{figure}[htb]
	\includegraphics[width=\textwidth]{NiAl_2D_K1e-7.png} 
	\caption{Heatmaps of concentrations over time calculated by conserved phase-field method from an initially randomly distributed concentration map with a K value of \(10^{-7}\).}
	\label{fig:NiAl_2D_heatmaps_K1e-7}
\end{figure}


%\listoffigures
\printbibliography



\end{document}
