
\input{preamble.tex}

\begin{document}

%
\begin{titlepage}
\begin{center}
\includegraphics[width=0.5\textwidth]{graphics/FAU_TechFak_EN_H_black.eps}

\LARGE Department Materials Science

\Large WW8: Materials Simulation

\LARGE \textbf{Practical: Phase-Field Method}

\Large Basics and Application in Materials Science



\vfil
\Large Leon Pyka (22030137)



\Large \textbf{Supervision: }
\end{center}

\thispagestyle{empty}
%
\end{titlepage}
%

\setcounter{page}{1}

\tableofcontents
\newpage
\section{Introduction}
Phase-field simulation is a versatile application in the toolbox of materials simulation. It is often used for simulations of phase transitions, dislocation evolution, fracture simulations etc. The following practicals aim is to get a practical introduction into the subject. In two separate tasks a 1D single-component solidification simulation, calculation of a bulk energy density coefficient and gradient energy density coefficient are going to be conducted. 

\section{Task 1: 1D Single-Component Solidification}

To simulate a 1D single-component solidification we utilize the following energy density:

\begin{equation}
	F = \int \bigl( f_{0} \phi^{2}(1- \phi)^{2}  + \frac{K_{\phi}}{2} \lvert \nabla \phi \rvert ^{2} \bigr) d \vec{r}
\end{equation}

As soldification is a a non-conservative process, the kinetics are governed by the Allen-Cahn equation (for the  homogeneous, isotropic case).

\begin{equation}
	\frac{\partial \phi}{\partial t} =-L \frac{\delta F}{\delta \phi} \label{eq:alle_cahn_homo_iso}
\end{equation}

Here the functional derivative \(\frac{\delta F}{\delta \phi}\) of an energy functional eq. \ref{eq:energy_functional_general} can be evaluated as eq. \ref{eq:func_derivative_general}.\\


\begin{subequations}
	\begin{align}
		F =& \int f(\vec{r}, \phi, \nabla \phi) d\vec{r} \label{eq:energy_functional_general} \\
		\frac{\delta F}{\delta \phi} =& \frac{\partial f}{ \partial \phi} - \nabla \cdot \frac{\partial f}{\partial (\nabla \phi)} \label{eq:func_derivative_general}
		\end{align}
\end{subequations}

Solving the functional derivative from eq. \ref{eq:alle_cahn_homo_iso} yields:

\begin{subequations}
	\begin{align}
		\frac{\delta F}{\delta \phi} =& 2 f_{0}\phi (1-\phi)^2 + 2 f_{0}\phi^{2} (\phi -1) - K_{\phi} \nabla^{2} \phi \Leftrightarrow \\
		\Leftrightarrow  & 2 f_0 (2\phi^{3} -\phi^{2} + \phi) - K_{\phi} \nabla^{2} \phi 
	\end{align}
\end{subequations}

Inserting the result into eq. \ref{eq:alle_cahn_homo_iso} results in:

\begin{equation}
	\frac{\partial \phi}{\partial t} = -L \bigl[ 2 f_0 (2\phi^{3} -\phi^{2} + \phi) - K_{\phi} \nabla^{2} \phi \bigr]
\end{equation}
%\listoffigures
\printbibliography

\end{document}
