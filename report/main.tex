
\input{preamble.tex}

\begin{document}

%
\begin{titlepage}
\begin{center}
\includegraphics[width=0.5\textwidth]{graphics/FAU_TechFak_EN_H_black.eps}

\LARGE Department Materials Science

\Large WW8: Materials Simulation

\LARGE \textbf{Practical: Phase-Field Method}

\Large Basics and Application in Materials Science



\vfil
\Large Leon Pyka (22030137)



\Large \textbf{Supervision: }
\end{center}

\thispagestyle{empty}
%
\end{titlepage}
%

\setcounter{page}{1}

\tableofcontents
\newpage
\section{Introduction}
Phase-field simulation is a versatile application in the toolbox of materials simulation. It is often used for simulations of phase transitions, dislocation evolution, fracture simulations etc. The following practicals aim is to get a practical introduction into the subject. In two separate tasks a 1D single-component solidification simulation and the calculation of a bulk energy density coefficient are going to be conducted. 


%\listoffigures
\printbibliography

\end{document}
